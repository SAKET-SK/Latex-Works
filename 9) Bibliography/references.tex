\documentclass[12pt]{article}
\usepackage{amsmath,cclicenses}
\title{Tutorial on Bibliography}
\author{Kannan Moudgalya \\ kannan@iitb.ac.in \\ \byncsa}
\date{\today}
\bibliographystyle{plain}

\begin{document}
\maketitle
\newpage
\section{Aryabhatta's Identity for Control Design}

Polynomial equations of the form
\begin{align*}
X(z)D(z) + Y(z)N(z) = C(z)
\end{align*}
arise frequently in control system design.  In the above equation,
$D(z)$, $N(z)$ and $C(z)$ are known polynomials and $X(z)$ and 
$Y(z)$ are unknowns, to be determined.  This equation is known as 
Diophantine equation \cite{vk79,tk80} and Aryabhatta's identity
\cite{mv85}.  A solution technique to this identity is presented in
\cite{cp82}.  Matlab and Scilab implementations of this solution are
available on the web \cite{kmm1-07}.

The textbook by \cite{kmm07} 
illustrates several control design
examples using Aryabhatta's identity.  The approach followed in this
book is explained in \cite{ms04,km06}.  In addition to handling
control design problems in conventional domains, this approach will 
be useful also for naturally discrete time problems that arise in
computing systems, see for example, \cite{mmr03,mrbm04,vs06}.


\bibliography{ref}
\end{document}
