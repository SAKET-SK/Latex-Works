% theme split
\usepackage{beamerthemesplit}

% theme shadow
\usepackage{beamerthemeshadow}

% logo
\logo{\includegraphics[height=1cm]{iitblogo.pdf}}

% sf family, bold font
\sffamily \bfseries

% running title and author information
[Presentation using \LaTeX\ and Beamer
\hspace{0.5cm}
\insertframenumber/\inserttotalframenumber]

[Kannan Moudgalya]

% equations - from spoken tutorial on equations
\begin{frame}
\frametitle{Equations:  from Spoken Tutorial on
  Equations} 
Let us start with the model of an inverted
pendulum:
\begin{align*}
\frac d{dt} 
\begin{bmatrix} x_1 \\ x_2 \\ x_3 \\
  x_4 \end{bmatrix} & =
\begin{bmatrix} 
0 & 0 & 1 & 0 \\
0 & 0 & 0 & 1 \\
0 & -\gamma & 0 & 0 \\
0 & \alpha & 0 & 0
\end{bmatrix}
\begin{bmatrix} x_1 \\ x_2 \\ x_3 \\
  x_4 \end{bmatrix} +
\begin{bmatrix} 0 \\ 0 \\ -\delta \\
  -\beta \end{bmatrix} \Delta\mu
%
\intertext{Proportional, integral, derivative
  controller is most popular in industry.  It has
  three tuning parameters: $K$, $\tau_i$ and
  $\tau_d$.  The integral mode includes the term
  $\int_0^t()dt$.}  
u(t) & = K \left\{ e(t)+\frac
  1{\tau_i}\int_0^te(t)dt + \tau_d\frac{de(t)}{dt}
\right\}
\end{align*}
\end{frame}

% change colour of one side of equation to blue

% simple animation
\begin{frame}
  \frametitle{From Spoken Tutorial on Letter
    Writing}
National Mission on Education through ICT, with an
outlay of Rs. 4,600 crore (\$ 1 billion): \pause
\begin{enumerate}
\item<+-|alert@+> Rs. 1,800 crore has been reserved for
  content generation and the rest to establish good
  connectivity in all 20,000 colleges and 200
  universities.
\item<+-|alert@+> Support for all good proposals,
  including those from private colleges.
\item<+-|alert@+> All products funded by this mission
  will be delivered as open source.
\item<+-|alert@+> Web based support through
  www.sakshat.ac.in.
\end{enumerate}
\end{frame}

% change the alerted colour to blue
\setbeamercolor{alerted text}{fg=blue}

% change the colour to brown and then to blue again

% including figures
\begin{frame}
\frametitle{Example of a Figure}
\begin{center}
\includegraphics[width=\linewidth]{iitb}
\end{center}
\end{frame}

% guidelines for including figures
\begin{frame}
\frametitle{Hints for Including Figures}
\begin{itemize}
\item<+-> Do not use floated environments in
  presentations 
  - \\ {\tt begin\{figure\}}, {\tt end\{figure\}}, 
  etc. 
\item<+-> Use {\tt includegraphics} directly
\item<+-> Do not include {\tt caption}, figure
  number, etc.
\item<+-> The audience will not remember figure
  numbers, any way
\item<+-> If you want to refer to a previously
  shown figure, show it again - do not refer to it
  by number
\end{itemize}
\end{frame}

% two columns
\begin{frame}
\frametitle{Two Columns}
\begin{minipage}[c]{0.45\textwidth}
\begin{enumerate}
\item<+-|alert@+> Rs. 1,800 crore has been reserved for
  content generation and the rest to establish good
  connectivity in all 20,000 colleges and 200
  universities.
\item<+-|alert@+> Support for all good proposals,
  including those from private colleges.
\end{enumerate}
\end{minipage} 
\hfill
\begin{minipage}[c]{0.45\textwidth}
\includegraphics[width=\linewidth]{iitb}
\end{minipage}
\end{frame}

% example of a Table
\begin{frame}
\frametitle{Table: From Spoken Tutorial on Tables} 
\begin{center}
\begin{tabular}{||l|c|c|c|r|}\hline
\multicolumn 2 {||c|}{Fruit details} & 
\multicolumn 3 {c|}{Cost calculations} \\ \hline
Fruit & Type & No. of units & cost/unit & cost (Rs.) \\
\hline 
Mango & Malgoa & 18 & 50 &  \\ \cline{2-4}
      & Alfonso & 2 & 300 & 1,500 \\ \hline
Jackfruit & Kolli Hills & 10 & 50 & 500 \\ \hline
Banana & Green & 10 & 20 & 200 \\ \hline
\multicolumn 4{||r|}{Total cost (Rs.)} & 2,200 \\
\hline 
\end{tabular}
\end{center}
\end{frame}

% guidelines for including Tables
\begin{frame}
\frametitle{Hints for Including Tables}
\begin{itemize}
\item<+-> Do not use floated environments in
  presentations 
  - \\ {\tt begin\{table\}}, {\tt end\{table\}}, 
  etc. 
\item<+-> Use {\tt tabular} directly
\item<+-> Do not include {\tt caption}, Table
  number, etc.
\item<+-> The audience will not remember Table
  numbers, any way
\end{itemize}
\end{frame}

% different animation in the previous slide

% convert document into handout

% using fragile with verbatim
\begin{frame}[fragile]
\frametitle{Example of Verbatim to Illustrate Scilab}
\begin{verbatim}
-->a = 1:5, b = 1:2:9
\end{verbatim}
{\color{blue}
\begin{verbatim}
 a  =
 !   1.    2.    3.    4.    5. !
 b  =
 !   1.    3.    5.    7.    9. !
\end{verbatim}}

\begin{verbatim}
-->a - 2
\end{verbatim}
{\color{blue}
\begin{verbatim}
ans  =
 ! - 1.    0.    1.    2.    3. !
\end{verbatim} }
\end{frame}

% where to get more information
\begin{frame}
\frametitle{Where to Get more Information on
  Beamer?} 
\begin{itemize}
\item<+-|alert@+> Authoritative source: user guide to the Beamer class
\item<+-|alert@+> Author: Till Tantau
\item<+-|alert@+> File name: 
  {\color{magenta} \tt beameruserguide.pdf} 
\item<+-|alert@+> email: tantau@users.sourceforge.net 
\item<+-|alert@+> At the time of creating this
  tutorial, available at \\
{\color{magenta} \small http://www.ctan.org/tex-archive/macros/latex/contrib/beamer/doc/beameruserguide.pdf}
\item<+-|alert@+> The web page for the Beamer
  project is
{\color{magenta} http://latex-beamer.sourceforge.net}
\end{itemize}
\end{frame}

% scroll user guide, copy one frame from there

% change back to presentation

% acknowledgement
\begin{frame}
\frametitle{Acknowledgement}
\begin{itemize}
\item Funding to create this tutorial has came from
  the \\ National Mission on Education through ICT
\item Web site for this mission:
  {\color{magenta} http://sakshat.ac.in}
\end{itemize}
\end{frame}

% feedback
\begin{frame}
\frametitle{Feedback}
\begin{itemize}
\item We have come to end of this tutorial
\item Thanks for joining 
\item Please give your feedback at kannan@iitb.ac.in
\end{itemize}
\end{frame}
